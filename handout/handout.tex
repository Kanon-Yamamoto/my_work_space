  \documentclass[a4j,twocolumn]{jsarticle}
  \usepackage[dvipdfmx]{graphicx}
  \usepackage{url}

  \setlength{\textheight}{275mm}
  \headheight 5mm
  \topmargin -30mm
  \textwidth 185mm
  \oddsidemargin -15mm
  \evensidemargin -15mm
  \pagestyle{empty}


  \begin{document}

  \title{デジタル版しおりの開発}
  \author{情報工学課程 \hspace{5mm} 37022463 \hspace{5mm} 西谷研究室 山本果音}
  \date{}

  \maketitle
\maketitle




\section{序論}
\label{sec:org1612ce1}

修学旅行などのイベントでは, 参加者に対して事前にしおりが配布されることが多い.
しおりは旅程の把握や所持品の確認, 集合時間の共有などを目的とした重要な情報媒体であり, 旅行の準備や当日の行動において欠かせない役割を果たしている.
近年では,スマートフォンの普及により, 紙媒体に代わってWebサービスを活用した旅行データの閲覧・共有の
ニーズが高まっており, 「旅しお」[1]のようなWebアプリも登場している.
しかし, これらのサービスにはいくつかの課題が存在する.

1つ目は, 旅行名や説明文に対するキーワード検索機能が存在しないため, ユーザーはスクロール操作によって目的のしおりを探す必要がある点である.

2つ目は, 地図表示機能がないため, 目的地の位置関係や全体像を視覚的に理解することが難しく, 利用者は外部ツールや手作業で補完しなければならない.

3つ目は, 移動時間の自動算出機能が存在しないため, 各目的地間の所要時間をユーザー自身が都度手動で調べる必要がある点である. これにより, 旅行全体のスケジュールを立てる際に手間がかかり, 計画の精度や効率性が低下する可能性がある.

これらの課題は, 旅行計画の際にユーザーにとって大きな障壁となっており, データの整理や視覚的な理解の低下が考えられる.
そこで本研究では, 視覚的な理解と効率的に旅行計画を行える「デジタル版しおり」として機能する旅行アプリの開発を目指す.

\section{開発手法}
\label{sec:org9f02e0a}
本研究では, 限られた開発期間内で効率的に成果を上げるために, 機能ごとに優先順位を設定し, それぞれについて計画・設計・実装・テストの工程を小刻みに繰り返すアジャイル手法を採用した.  

開発環境としてはDjangoを選定した.
Djangoは,PythonベースのWebフレームワークである.
Djangoの利点は以下の3つである[2].
\begin{enumerate}
\item 認証,管理画面,データベース操作などが最初から組み込まれているため,追加設定なしで迅速な開発が可能.
\item データベースの設計変更やマイグレーションを含む複雑なデータベース操作も,全てPythonで記述できるため学習コストが低い.
\item セキュリティ対策を実装しなくても,フレームワーク自体が標準で安全性を確保する仕組みを備えている.
\end{enumerate}
また, 実装にはPython, HTML, JavaScript, CSSを使用し,UIライブラリとしてはBootstrap, Font Awesomeを採用した.

\section{結果}
\label{sec:orgf4aac65}
本研究では主に7つのアプリ機能を実装した.

\begin{figure}[htbp]
\centering
\includegraphics[width=6cm]{./figs/app_1.png}
\caption{\label{fig:org7c3fe51}アプリ機能.}
\end{figure}

本アプリでの旅行詳細共有までの流れとして, まず新規アカウント登録後にログイン画面へ遷移し, 旅行一覧画面で旅行を作成する. その後, 旅行詳細画面において地図表示, 観光地登録, 移動時間の自動算出, 訪問順序の入れ替え, タブ切り替えなどの旅程編集を行い, 最終的に共有リンクを生成して家族や友人と旅行情報を共有できる.

\begin{figure}[htbp]
\centering
\includegraphics[width=6cm]{./figs/app_2.png}
\caption{旅行詳細共有までの一連の流れ.}
\end{figure}

さらに, 旅行準備を効率的に進められるよう, チェックリストに関する複数の機能を実装した.
チェックリストは旅行ごとに作成・編集でき, チェック状態の保存や再利用にも対応している. これにより, 忘れ物対策や準備作業の整理が容易となり, 旅行前の負担軽減に寄与する.

\small\setlength\baselineskip{10pt}
\begin{thebibliography}{9}

\bibitem{旅しお} 旅しお,\url{https://tabisio.com/},(2025/09/05 accessed).
\bibitem{Django-site}Django,\url{https://www.djangoproject.com/},(2025/09/05 accessed).
\bibitem{GitHub Copilot} GitHub Inc,\url{https://github.com/},(2025/09/05 accessed).
\end{thebibliography}
\end{document}
